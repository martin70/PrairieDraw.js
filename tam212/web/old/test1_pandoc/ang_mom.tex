\documentclass{article}
\usepackage{amsmath,amssymb,amsthm}
\usepackage{hyperref}
\usepackage[margin=3cm]{geometry}

\hypersetup{colorlinks=true, linkcolor=blue,  anchorcolor=blue,
citecolor=blue, filecolor=blue, menucolor=blue, pagecolor=blue,
urlcolor=blue}

\usepackage{manfnt}

\makeatletter
\def\hang{\hangindent\parindent}
\def\d@nger{\medbreak\begingroup\clubpenalty=10000
  \def\par{\endgraf\endgroup\medbreak} \noindent\hang\hangafter=-2
  \hbox to0pt{\hskip-\hangindent\dbend\hfill}}
\outer\def\danger{\d@nger}
\makeatother

\renewcommand{\vec}[1]{\boldsymbol{#1}}
\renewcommand{\matrix}[1]{\boldsymbol{#1}}

\begin{document}

\section{Angular Momentum in 3D}

The angular momentum of a body describes its tendency to continue
rotational motion, analogous to the way that
\href{linear_momentum}{linear momentum} describes the tendency to
continue translational motion. Angular momentum in 3D is a vector
quantity $\vec{H}$, while \href{angular_momentum_2d}{in 2D} it is a
scalar $H$. The total angular momentum of several bodies is the sum of
their individual angular momenta. Angular momentum changes due to the
action of \href{moment}{moments} on the body, as described by
\href{newtons_eqn_moment}{Newton's moment equation}.

\subsection{Definition of angular momentum in 3D}

\subsubsection{Point masses}

(\href{angular_momentum_point_masses_video}{Video explanation})

The angular momentum of a \href{point_mass}{point mass} $P$ about some
\href{origin}{origin} $O$ is
\begin{equation}
  \label{eqn:ang_mom_3d_point_mass}
  \vec{H}_O = \vec{r}_{OP} \times m \vec{v} = \vec{r}_{OP} \times \vec{p},
\end{equation}
where $\vec{r}_{OP}$ is the \href{position_vector}{position vector} of
the point mass, $m$ is the mass, $\vec{v}$ is the
\href{velocity}{velocity}, and $\vec{p} = m \vec{v}$ is the
\href{linear_momentum}{linear momentum}.

\subsubsection{Rigid bodies}

(\href{angular_momentum_rigid_bodies_video}{Video explanation})

The angular momentum of a \href{rigid_body}{rigid body} about its
center of mass $C$ depends on its \href{angular_velocity}{angular
  velocity} $\omega$ and \href{moment_of_inertia_matrix}{moment of
  inertia matrix} $\matrix{I}_C$, with
\begin{equation}
  \label{eqn:ang_mom_3d_rigid_body}
  \vec{H}_C = \matrix{I}_C \vec{\omega}.
\end{equation}

\subsection{Properties of angular momentum}

\subsubsection{Dependence on reference point}

The angular momentum of a body depends on the reference
point. We write $\vec{H}_C$ to indicate that this angular momentum is
with respect to the point $C$.

Consider the example below, with a point (the black dot) moving in
uniform circular motion. The red arrows show the angular momentum
about nine different reference points. We see that $H_e$ has a
constant value in the counter-clockwise direction, while the other
angular momenta vary over time and even reverse direction. This is
despite the fact that the point is always moving with a constant
angular velocity about the center $e$. Note that the largest angular
momentum is about the furthest-away point.

Another example is shown below, this time with the point moving in a
square with variable speed. Again we note that the angular momentum is
larger about far-away points. Also note that many of these angular
momenta are equal, as the projection of the linear momentum onto the
perpendicular direction gives the same value.

\subsubsection{Adding angular momenta}

(\href{angular_momentum_adding_video}{Video explanation}) 

If we have two bodies, whether connected or not, then the total
angular momentum is the sum
\begin{equation}
  \label{eqn:ang_mom_addition}
  \vec{H}_O^{\rm tot} = \vec{H}_O^1 + \vec{H}_O^2
\end{equation}
of the angular momenta $\vec{H}_O^1$ and $\vec{H}_O^2$ of bodies 1 and
2.

\danger Angular momenta can only be added together if they are both
about the same reference point. It makes no sense to write
$\vec{H}_O^1 + \vec{H}_C^2$, for example.

\subsubsection{Relationship between angular momentum in 2D and 3D}

(\href{angular_momentum_2d_vs_3d_video}{Video explanation})

\subsection{Calculating angular momentum}

\subsection{Solving problems using angular momentum}

\subsection{See also}

\begin{itemize}
\item \href{angular_momentum_2d}{Angular momentum in 2D}
\item \href{angular_momentum_rigid_body_derivation}{Derivation of the
    rigid body angular momentum equation}
\item \href{moment}{Moment}
\item \href{linear_momentum}{Linear momentum}
\item \href{newtons_eqn_moment}{Newton's moment equation}
\end{itemize}

\subsection{External links}

\begin{itemize}
\item Wikipedia page:
  \href{http://en.wikipedia.org/wiki/Angular_momentum}{Angular
    momentum}
\end{itemize}

\subsection{Example problems}

\begin{itemize}
\item \href{hw5-2_q2}{Homework 5.2, Question 2}
  (\href{hw5-2_q2_soln}{Solution},
  \href{hw5-2_q2_explain}{Explanation})
\end{itemize}

\subsection{Applications}

\begin{itemize}
\item \href{transition_curves}{Transition curves for highway design}
\item \href{satellite_despin}{Satellite despinning}
\end{itemize}

\subsection{Prerequisite Topics}

\begin{itemize}
\item \href{point_mass_kinematics}{Point mass kinematics}
\item \href{rigid_body_kinematics_3d}{Rigid body kinematics in 3D}
\item \href{angular_momentum_2d}{Angular momentum in 2D}
\end{itemize}

\subsection{Follow-on Topics}

\begin{itemize}
\item \href{newtons_eqn_moment}{Newton's moment equation}
\end{itemize}

\end{document}
